\documentclass[pdf]{prosper}
\usepackage{amsmath}
\DeclareSymbolFont{AMSb}{U}{msb}{m}{n}
\DeclareMathSymbol{\natural}{\mathbin}{AMSb}{"4E}
\DeclareMathSymbol{\integer}{\mathbin}{AMSb}{"5A}
\DeclareMathSymbol{\real}{\mathbin}{AMSb}{"52}
\DeclareMathSymbol{\rational}{\mathbin}{AMSb}{"51}
\DeclareMathSymbol{\I}{\mathbin}{AMSb}{"49}
\DeclareMathSymbol{\complex}{\mathbin}{AMSb}{"43}


\title{Darcs versus Git}
\author{Miles Gould}
\institution{University of Edinburgh}
\begin{document}
\maketitle
\begin{slide}{Similarities}
\begin{itemize}
\item Both highly distributed
\item Both allow rewriting of unpublished history
\item Many features borrowed from each other
\item But under the hood, totally different
\end{itemize}
\end{slide}
\begin{slide}{Git}
\begin{itemize}
\item Written by some Finnish dude in 2005 as a side-project
\item C/shell/Perl/Python/...
\item content-addressable filesystem
\begin{itemize}
	\item everything indexed by hash of contents
	\item a version is a tree of blobs
	\item a commit is a tree, pointers to parents, metadata
	\item diffs calculated as needed
\end{itemize}
\item Inspired by Monotone (but faster)
\item History is a DAG of commits (snapshots).
\end{itemize}
\end{slide}
\begin{slide}{Darcs}
\begin{itemize}
\item Written by physicist David Roundy in 2003
\item Initially C++, rewritten in Haskell
\item A repository is an \emph{unordered collection of patches}
\item A patch is an \emph{abstract atomic change}
\item To apply a patch to a repo, must calculate a concrete \emph{effect}
\item Potential for high-level patch types
\item Patches may \emph{depend} on other patches
\end{itemize}
\end{slide}
\end{document}
